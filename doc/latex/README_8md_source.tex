\hypertarget{README_8md_source}{\section{R\+E\+A\+D\+M\+E.\+md}
}

\begin{DoxyCode}
00001 \textcolor{preprocessor}{# URY playd}
00002 
00003 URY \hyperlink{classplayd}{playd} (`\hyperlink{classplayd}{playd}` \textcolor{keywordflow}{for} \textcolor{keywordtype}{short}) is a minimal C++ audio player \textcolor{keyword}{using} [libsox][]
00004 and [PortAudio][], developed by [University Radio York][] (URY) and
00005 designed to be composable into bigger systems.
00006 
00007 All code developed \textcolor{keywordflow}{for} `\hyperlink{classplayd}{playd}` is licenced under the [MIT licence][]
00008 (see LICENCE.txt).  Some code is taken from the [PortAudio][] project
00009 (see LICENCE.portaudio).
00010 
00011 ## Usage
00012 
00013 `\hyperlink{classplayd}{playd} DEVICE-ID [ADDRESS] [PORT]`
00014 
00015 * Invoking `\hyperlink{classplayd}{playd}` with no arguments lists the various device IDs
00016   available to it.
00017 * Full protocol information is available on the GitHub wiki.
00018 * On POSIX systems, see the enclosed man page.
00019 
00020 `\hyperlink{classplayd}{playd}` understands the following commands via its TCP/IP interface:
00021 
00022 * `load \textcolor{stringliteral}{"/full/path/to/file"}` — Loads /full/path/to/file \textcolor{keywordflow}{for} playback;
00023 * `eject` — Unloads the current file;
00024 * `play` — Starts playback;
00025 * `stop` — Stops (pauses) playback;
00026 * `seek 1m` — Seeks one minute into the current file.  Units supported include
00027   `h`, `m`, `s`, `ms`, `us` (micros), with `us` assumed if no unit is given.
00028 * `quit` — Closes `\hyperlink{classplayd}{playd}`.
00029 
00030 \textcolor{preprocessor}{### Sending commands manually}
00031 
00032 To connect directly to `\hyperlink{classplayd}{playd}` and issue commands to it, you can use
00033 [netcat][]:
00034 
00035 ```sh
00036 \textcolor{preprocessor}{# If you specified [ADDRESS] or [PORT], replace localhost and 1350 respectively.}
00037 $ nc localhost 1350
00038 ```
00039 
00040 On Windows, \textcolor{keyword}{using} [PuTTY][] in \_raw mode\_ (\_\_not\_\_ Telnet mode)
00041 with \_Implicit CR in every LF\_ switched on in the \_Terminal\_ options should
00042 work.
00043 
00044 \_\_Do \_not\_ use a Telnet client (or PuTTY in telnet mode)!\_\_  `\hyperlink{classplayd}{playd}` will
00045 \textcolor{keywordflow}{do} weird things in the presence of Telnet-isms.
00046 
00047 ## Features
00048 
00049 * Plays anything [libsox][] can play (in practice, more esoteric formats might
00050   not work)
00051 * Seek (microseconds, seconds, minutes etc)
00052 * Frequently announces the current position
00053 * TCP/IP \textcolor{keyword}{interface }with text protocol
00054 * Deliberately not much \textcolor{keywordflow}{else}
00055 
00056 \textcolor{preprocessor}{## Philosophy}
00057 
00058 * Do one thing and \textcolor{keywordflow}{do} it well
00059 * Be hackable
00060 * Favour simplicity over performance
00061 * Favour simplicity over features
00062 * Let other programs handle the shinies
00063 
00064 \textcolor{preprocessor}{## Compilation}
00065 
00066 \textcolor{preprocessor}{### Requirements}
00067 
00068 * [libsox][] (1.14.1)
00069 * [libuv][] (0.11.29)
00070 * [PortAudio][] (19\_20140130)
00071 * A C++11 compiler (recent versions of [clang][], [gcc][], and Visual Studio
00072   work)
00073 
00074 Certain operating systems may need additional dependencies; see the OS-specific
00075 build instructions below.
00076 
00077 ### POSIX (GNU/Linux, BSD, OS X)
00078 
00079 `\hyperlink{classplayd}{playd}` comes with a GNU-compatible Makefile that can be used both to
00080 make and install.
00081 
00082 To use the Makefile, you\textcolor{stringliteral}{'ll need [GNU Make][] and `pkg-config` (or equivalent),}
00083 \textcolor{stringliteral}{and pkg-config packages for PortAudio, libsox and libuv.  We'}ve tested building
00084 \hyperlink{classplayd}{playd} on Gentoo, FreeBSD 10, and OS X, but other POSIX-style operating systems
00085 should work.
00086 
00087 Using the Makefile is straightforward:
00088 
00089 * Ensure you have the dependencies above;
00090 * Read the `Makefile`, to see \textcolor{keywordflow}{if} any variables need to be overridden \textcolor{keywordflow}{for} your
00091   environment;
00092 * Run `make` (or whatever GNU Make is called on your OS; in FreeBSD, \textcolor{keywordflow}{for}
00093   example, it\textcolor{stringliteral}{'d be `gmake`), and, optionally, `sudo make install`.}
00094 \textcolor{stringliteral}{  The latter will globally install playd and its man page.}
00095 \textcolor{stringliteral}{}
00096 \textcolor{stringliteral}{#### OS X}
00097 \textcolor{stringliteral}{}
00098 \textcolor{stringliteral}{All dependencies are available in [homebrew][] - it is highly recommended that}
00099 \textcolor{stringliteral}{you use it!}
00100 \textcolor{stringliteral}{}
00101 \textcolor{stringliteral}{### FreeBSD (10+)}
00102 \textcolor{stringliteral}{}
00103 \textcolor{stringliteral}{FreeBSD 10 and above come with `clang` 3.3 as standard, which should be able to}
00104 \textcolor{stringliteral}{compile `playd`.  `gcc` is available through the FreeBSD Ports Collection}
00105 \textcolor{stringliteral}{and package repositories.}
00106 \textcolor{stringliteral}{}
00107 \textcolor{stringliteral}{You will need `gmake`, as `Makefile` is incompatible with BSD make.  Sorry!}
00108 \textcolor{stringliteral}{}
00109 \textcolor{stringliteral}{All of `playd`'}s dependencies are available through both the FreeBSD Ports
00110 Collection and standard \textcolor{keyword}{package }repository.  (The FreeBSD port for PortAudio
00111 doesn't build C++ bindings, but we bundle them anyway.)  To install them as
00112 packages:
00113 
00114 ```
00115 root:/ # pkg install gmake sox libuv portaudio2 pkgconf
00116 ```
00117 
00118 Then, run `gmake` (\_\_not\_\_ `make`), and, optionally, `gmake install` to install
00119 `playd` (as root):
00120 
00121 ```
00122 user:~/ % gmake
00123 root:~/ # gmake install
00124 ```
00125 
00126 ### Windows
00127 
00128 #### Visual Studio
00129 
00130 \_For more information, see `README.VisualStudio.md`.\_
00131 
00132 playd **can** be built with Visual Studio (tested with 2013 Premium), but
00133 you will need to source and configure the dependencies manually.  A Visual
00134 Studio project is provided, but will need tweaking for your environment.
00135 
00136 #### MinGW
00137 
00138 We haven't managed ourselves, but assuming you can build all the dependencies,
00139 (libsox is the difficult one), it should work fine.
00140 
00141 ### PortAudio C++ Bindings
00142 
00143 If you have the PortAudio C++ bindings available, those may be used in place of
00144 the bundled bindings.  This will happen automatically when using the Makefile,
00145 if the C++ bindings are installed as a pkg-config package.
00146 
00147 \_\_Visual Studio users:\_\_ The Visual Studio 7.1 project supplied in the
00148 PortAudio source distribution for building the C++ bindings
00149 (`\(\backslash\)bindings\(\backslash\)cpp\(\backslash\)build\(\backslash\)vc7\_1\(\backslash\)static\_library.vcproj`) should work.  If not, then
00150 use the bundled bindings.
00151 
00152 ## Q&A
00153 
00154 ### Why does this exist?
00155 
00156 It was originally written as an experiment when coming up with a new playout
00157 system for [University Radio York][].
00158 
00159 ### Why is it named `playd`?
00160 
00161 It's short for \_\_play\_\_er \_\_d\_\_aemon.
00162 
00163 ### Can I contribute?
00164 
00165 Certainly!  We appreciate any and all pull requests in accordance with our
00166 philosophy.
00167 
00168 [clang]:                 http:\textcolor{comment}{//clang.llvm.org/}
00169 [gcc]:                   https:\textcolor{comment}{//gcc.gnu.org/}
00170 [GNU Make]:              https:\textcolor{comment}{//www.gnu.org/software/make/}
00171 [Homebrew]:              http:\textcolor{comment}{//brew.sh}
00172 [libsox]:                http:\textcolor{comment}{//sox.sourceforge.net/}
00173 [libuv]:                 https:\textcolor{comment}{//github.com/joyent/libuv}
00174 [MIT licence]:           http:\textcolor{comment}{//opensource.org/licenses/MIT}
00175 [netcat]:                http:\textcolor{comment}{//nc110.sourceforge.net/}
00176 [PortAudio]:             http:\textcolor{comment}{//www.portaudio.com/}
00177 [PuTTY]:                 http:\textcolor{comment}{//www.chiark.greenend.org.uk/~sgtatham/putty/}
00178 [University Radio York]: http:\textcolor{comment}{//ury.org.uk}
\end{DoxyCode}
